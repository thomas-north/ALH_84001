Caption for chromite_cooling_ALH84001.mp4

The video demonstrates the temperature evolution of the whole sample immediately following the passage of a planar shockwave. 
The simulation displayed is the 33 GPa 'reference' simulation. Temperature arrays were calculated according to the heat equation
described in equation (1) in the article, while the timestep dt between calculations was $\Delta t$/5, where $\Delta t$ was calculated
according to equation (2) in the article.

(a) displays the temperature map of all materials within the sample and is colored relative to the equilibrium temperature (186 C).

In pane (b) we have colored all non-chromite material gray to show a comprehensive overview of the thermal experience in chromites 
throughout the sample. In doing so we highlight the spatially heterogeneous nature of the heating where chromites in shear zones 
(such as the chromite circled) experience secondary maxima, while the average chromite temperature remains below the thermal equilibrium
temperature.

(c) describes the evolution of both the peak and average temperature recorded in the circled chromite, compared to the average temperature 
recorded in all chromites.

(d) is a close-up view of the circled chromite. We observe that the fringes connected to the shear zone in dunite are heated most of all,
while the center heats up over a period of milliseconds. 
